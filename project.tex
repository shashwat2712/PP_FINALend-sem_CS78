\documentclass{article}
\usepackage[utf8]{inputenc}
\usepackage{amsmath}
\usepackage{graphicx}
\usepackage{float}


\title{\textbf{Mini Project Report} }
\author{\textbf{Shashwat Sai Vyas}\\
0801CS211078}
\date{November 2022}

\begin{document}

\maketitle
\begin{center}
\title{\textbf{\large JEE Management Interface}}\\
\textbf{submitted to Sir Surendra Gupta}
\end{center}
\bigskip
\\
\begin{center}
\textbf{{\Large Index}}
\vspace{5mm} %5mm vertical space
\end{center}
    \begin{itemize}
       \item{\large  Objective}
       \item{\large  Overview}
       \item {\large Function Description}
       \item {\large Profiler Report}
       \item {\large GDB activities }
       \item {\large Git activities }
       \item { \large Source code with comments}
       \item {\large Output Screenshots}

    \end{itemize}

\par\noindent\rule{\textwidth}{0.4pt}

\pagebreak
\end{minipage}
}
\begin{center}
\textbf{{\Large Objective}}
\end{center}
    \begin{itemize}
       \item {\large The objective of the given program is to make a candidate management interface for the authorities.
       Through this program NTA or the JEE conducting authorites will easily manage the data and selection procedure.\\
       \item Our main focus is design a unique candidate management System that will improve Data
       maagement in Exam conduction and enhance experience for both Students and the Adminstartion authorities\\}

       \item{\large Various funtions are designed in the program which caters to a specific task}
    \end{itemize}

\end{minipage}
}
\par\noindent\rule{\textwidth}{0.4pt}


\end{minipage}
}
\vspace{1cm}
\begin{center}
\textbf{{\Large Overview}}
\end{center}
\begin{itemize}
    \item {\large The complete project is written in c language and contains 10 functions, also it uses various c concepts like file handling,structures etc to enhance the working of the program  
    \item For profiling task gprof tool was used.
    \item For Debugging purpose 
    \item For report creation of the mini project latex is used.
    }
\end{itemize}

\end{minipage}
}
\par\noindent\rule{\textwidth}{0.4pt}

\pagebreak

\bigskip
\\
 %5mm vertical space

\begin{center}
\textbf{{\Large Functions Description}}
\end{center}
\vspace{5mm}
{\large This program contains 10 functions and each function perform a specific task }
    \begin{enumerate}
       \item {\large take candidate info }
       \item {\large percentile calculator }
       \item {\large JEE log }
       \item {\large find candidate }
       \item {\large total }
       \item {\large advance qualification }
       \item {\large modify }
       \item {\large main }
       \item {\large display }
       \item {\large record the entry }
    \end{enumerate}
\par\noindent\rule{\textwidth}{0.4pt}
\vspace{5mm}
{\Large \textbf{1.take candidate info():}} \\
\\
{\large  Function to add the candidates along with all thier details
 It takes different input details of the candidate\\
 \vspace{5mm}
Following are the fields of details input by this function : \\
 \begin{itemize}
    \item name
    \item father name
    \item mother name
    \item roll
    \item chemistry marks
    \item physics marks
    \item maths marks
    \item jee rank
    \item all india rank
 \end{itemize}
 return type: void  \hspace{5mm}  Arguments: Nil\\
 time :\\
 It uses a series scanf functions to take input from the users\\
 Also gets function is also used to take input\\
 It also increments the indexing variable of the array of the structure at the last
 
 }
 \vspace{5mm}
{\Large \textbf{2. percentile calculator()}} \\
\\
{\large Program to find the percentile of the candidate\\
the function uses a simple algorithm which converts the rank to the percentile\
 \vspace{5mm}

return type: double  \hspace{5mm}  Arguments: takes integer rank\\
   time :\\
It takes an integer rank as input and converts it into the double percentile.
So basically it takes rank and gives the percentile back.
 }
 
  \vspace{5mm}
{\Large \textbf{3. JEE log()}} \\
\\
{\large Function for entering jee conduction details\\
also this function throws a brief introduction about the JEE examination.\\
 \vspace{5mm}

return type: int(numbers of students enrolled)  \hspace{5mm}  Arguments:Nil\\
   time :\\
It basically takes the input of the number of the candidates appearing for jee this year and then returns it, because it is needed in the percentile calculation task and also in finding the qualification status of the JEE advance exam of that candidate.Also it gives a brief intro about the exam conduction and JEE.
 }
 
   \vspace{5mm}
{\Large \textbf{4. find candidate()}} \\
\\
{\large Function to find the candidate by the roll number\\
This function will take a roll number as input and return all the details of that candidate.
 \vspace{5mm}

return type: Void  \hspace{5mm}  Arguments: Nil\\
   time :\\
It takes the roll number of the candidate and finds out all the details of that candidate including the percentile,name,etc of that candidate
 }
 
   \vspace{5mm}
{\Large \textbf{5.total() :}} \\
\\
{\large This is a simple program which takes three numbers as input returns their sum\\
 \vspace{5mm}

return type: int  \hspace{5mm}  Arguments: takes 3 integer inputs\\
   time :\\
 }
 
   \vspace{5mm}
{\Large \textbf{6. advance qualification()}} \\
\\
{\large This function checks the qualification status of a candidate for advance exam:
\\The function takes the roll number as inputs and returns advanced qualification status on the basis of category of the candidate.
 \vspace{5mm}

return type: void  \hspace{5mm}  Arguments: Nil\\
   time :\\
Contains many conditional statements for proper checking of the category of the candidate and according cutoffs.It takes roll number of the candidate as input and returns the qualification status of that candidate for the jee advance exam.
 }
 
   \vspace{5mm}
{\Large \textbf{7. modify()}} \\
\\
{\large Function to modify the candidate details\\
This function also takes roll number as inputs and is used to modify the details of that student.
 \vspace{5mm}

return type: void  \hspace{5mm}  Arguments: Nil\\
   time :\\
Every field of the candidate details can be reassigned using this function.
 }
 
   \vspace{5mm}
{\Large \textbf{8. main()}} \\
\\
{\large The main function is responsible for calling all the functions according to the user needs

return type: int  \hspace{5mm}  Arguments: Nil\\
   time :\\
an infinite loop is run here to continuously take inputs from the user untill the user wants 
also uses switch case to implemet the options given to the user for different tasks.
 
    \vspace{5mm}
{\Large \textbf{9. display()}} \\
\\
{\large Function to display the details after adding the candidate info\\
It is specifically designed with correct numbers of the identations to give a table like look
 \vspace{5mm}

return type: void  \hspace{5mm}  Arguments: Nil\\
   time :\\
Displays all the info of the candidate in a eye catchy format.
 }
 
    \vspace{5mm}
{\Large \textbf{10.  record the entry()}} \\
\\
{\large The below function uses file handling \\
and prints the data of the candidate into a seperate text file (emp.txt)
 \vspace{5mm}

return type: void  \hspace{5mm}  Arguments: Nil\\
   time :\\
It basically helps in keeping the record of the data and easy handling of the data using a seperate file.
 }
 
\end{minipage}
}
\vspace{5mm}
\par\noindent\rule{\textwidth}{0.4pt}

\pagebreak

\vspace{5mm}
\begin{center}
{\Large\textbf{Profiling Report}}
\end{center}
    
\vspace{5mm}
\par\noindent\rule{\textwidth}{0.4pt}


{\textbf The profiling report of the program by the gprof is here }
\begin{verbatim}
    Flat profile:

Each sample counts as 0.01 seconds.
 no time accumulated

  %   cumulative   self              self     total           
 time   seconds   seconds    calls  Ts/call  Ts/call  name    
  0.00      0.00     0.00        3     0.00     0.00  percentile_calculator
  0.00      0.00     0.00        1     0.00     0.00  advance_qualification
  0.00      0.00     0.00        1     0.00     0.00  display
  0.00      0.00     0.00        1     0.00     0.00  find_candidate
  0.00      0.00     0.00        1     0.00     0.00  jee_log
  0.00      0.00     0.00        1     0.00     0.00  modify
  0.00      0.00     0.00        1     0.00     0.00  record_the_entry
  0.00      0.00     0.00        1     0.00     0.00  take_candidate_info
  0.00      0.00     0.00        1     0.00     0.00  total

 %         the percentage of the total running time of the
time       program used by this function.

cumulative a running sum of the number of seconds accounted
 seconds   for by this function and those listed above it.

 self      the number of seconds accounted for by this
seconds    function alone.  This is the major sort for this
           listing.

calls      the number of times this function was invoked, if
           this function is profiled, else blank.

 self      the average number of milliseconds spent in this
ms/call    function per call, if this function is profiled,
	   else blank.

 total     the average number of milliseconds spent in this
ms/call    function and its descendents per call, if this
	   function is profiled, else blank.

name       the name of the function.  This is the minor sort
           for this listing. The index shows the location of
	   the function in the gprof listing. If the index is
	   in parenthesis it shows where it would appear in
	   the gprof listing if it were to be printed.

Copyright (C) 2012-2022 Free Software Foundation, Inc.

Copying and distribution of this file, with or without modification,
are permitted in any medium without royalty provided the copyright
notice and this notice are preserved.

		     Call graph (explanation follows)


granularity: each sample hit covers 4 byte(s) no time propagated

index % time    self  children    called     name
                0.00    0.00       1/3           take_candidate_info [8]
                0.00    0.00       1/3           display [3]
                0.00    0.00       1/3           main [15]
[1]      0.0    0.00    0.00       3         percentile_calculator [1]
-----------------------------------------------
                0.00    0.00       1/1           main [15]
[2]      0.0    0.00    0.00       1         advance_qualification [2]
-----------------------------------------------
                0.00    0.00       1/1           take_candidate_info [8]
[3]      0.0    0.00    0.00       1         display [3]
                0.00    0.00       1/1           total [9]
                0.00    0.00       1/3           percentile_calculator [1]
-----------------------------------------------
                0.00    0.00       1/1           main [15]
[4]      0.0    0.00    0.00       1         find_candidate [4]
-----------------------------------------------
                0.00    0.00       1/1           main [15]
[5]      0.0    0.00    0.00       1         jee_log [5]
-----------------------------------------------
                0.00    0.00       1/1           main [15]
[6]      0.0    0.00    0.00       1         modify [6]
-----------------------------------------------
                0.00    0.00       1/1           main [15]
[7]      0.0    0.00    0.00       1         record_the_entry [7]
-----------------------------------------------
                0.00    0.00       1/1           main [15]
[8]      0.0    0.00    0.00       1         take_candidate_info [8]
                0.00    0.00       1/1           display [3]
                0.00    0.00       1/3           percentile_calculator [1]
-----------------------------------------------
                0.00    0.00       1/1           display [3]
[9]      0.0    0.00    0.00       1         total [9]
-----------------------------------------------

 This table describes the call tree of the program, and was sorted by
 the total amount of time spent in each function and its children.

 Each entry in this table consists of several lines.  The line with the
 index number at the left hand margin lists the current function.
 The lines above it list the functions that called this function,
 and the lines below it list the functions this one called.
 This line lists:
     index	A unique number given to each element of the table.
		Index numbers are sorted numerically.
		The index number is printed next to every function name so
		it is easier to look up where the function is in the table.

     % time	This is the percentage of the `total' time that was spent
		in this function and its children.  Note that due to
		different viewpoints, functions excluded by options, etc,
		these numbers will NOT add up to 100%.

     self	This is the total amount of time spent in this function.

     children	This is the total amount of time propagated into this
		function by its children.

     called	This is the number of times the function was called.
		If the function called itself recursively, the number
		only includes non-recursive calls, and is followed by
		a `+' and the number of recursive calls.

     name	The name of the current function.  The index number is
		printed after it.  If the function is a member of a
		cycle, the cycle number is printed between the
		function's name and the index number.


 For the function's parents, the fields have the following meanings:

     self	This is the amount of time that was propagated directly
		from the function into this parent.

     children	This is the amount of time that was propagated from
		the function's children into this parent.

     called	This is the number of times this parent called the
		function `/' the total number of times the function
		was called.  Recursive calls to the function are not
		included in the number after the `/'.

     name	This is the name of the parent.  The parent's index
		number is printed after it.  If the parent is a
		member of a cycle, the cycle number is printed between
		the name and the index number.

 If the parents of the function cannot be determined, the word
 `<spontaneous>' is printed in the `name' field, and all the other
 fields are blank.

 For the function's children, the fields have the following meanings:

     self	This is the amount of time that was propagated directly
		from the child into the function.

     children	This is the amount of time that was propagated from the
		child's children to the function.

     called	This is the number of times the function called
		this child `/' the total number of times the child
		was called.  Recursive calls by the child are not
		listed in the number after the `/'.

     name	This is the name of the child.  The child's index
		number is printed after it.  If the child is a
		member of a cycle, the cycle number is printed
		between the name and the index number.

 If there are any cycles (circles) in the call graph, there is an
 entry for the cycle-as-a-whole.  This entry shows who called the
 cycle (as parents) and the members of the cycle (as children.)
 The `+' recursive calls entry shows the number of function calls that
 were internal to the cycle, and the calls entry for each member shows,
 for that member, how many times it was called from other members of
 the cycle.

Copyright (C) 2012-2022 Free Software Foundation, Inc.

Copying and distribution of this file, with or without modification,
are permitted in any medium without royalty provided the copyright
notice and this notice are preserved.

Index by function name

   [2] advance_qualification   [5] jee_log                 [7] record_the_entry
   [3] display                 [6] modify                  [8] take_candidate_info
   [4] find_candidate          [1] percentile_calculator   [9] total
\end{verbatim}

\pagebreak

\vspace{5mm}
\begin{center}
{\Large\textbf{GDB activities}}
\end{center}
\vspace{5mm}
\par\noindent\rule{\textwidth}{0.4pt}
\begin{figure}
    \centering
    \includegraphics[height=15cm, width=20cm]{image_11.png}
    \caption{Initiating the breakpoints}{H}
\end{figure}
\par\noindent\rule{\textwidth}{0.4pt}
\begin{figure}
    \centering
    \includegraphics[height=5cm, width=15cm]{image_12.png}
    \caption{initiating the breakpoints}
\end{figure}
\par\noindent\rule{\textwidth}{0.4pt}
\begin{figure}
    \centering
    \includegraphics[height=10cm, width=20cm]{image_13.png}
    \caption{initiating the watch points}{H}
\end{figure}
\par\noindent\rule{\textwidth}{0.4pt}
\begin{figure}
    \centering
    \includegraphics[height=15cm, width=20cm]{image_14.png}
    \caption{listing and initiating the breakpoints}{H}
\end{figure}
\par\noindent\rule{\textwidth}{0.4pt}
\begin{figure}
    \centering
    \includegraphics[height=15cm, width=20cm]{image_15.png}
    \caption{gdb activities}{H}
\end{figure}
\par\noindent\rule{\textwidth}{0.4pt}
\begin{figure}
    \centering
    \includegraphics[height=15cm, width=20cm]{image_16.png}
    \caption{gdb activities}{H}
\end{figure}
\par\noindent\rule{\textwidth}{0.4pt}
\begin{figure}
    \centering
    \includegraphics[height=5cm, width=20cm]{image_17.png}
    \caption{gdb activities}{H}
\end{figure}
\pagebreak
\par\noindent\rule{\textwidth}{0.4pt}
\begin{figure}
    \centering
    \includegraphics[height=15cm, width=20cm]{image_18.png}
    \caption{gdb activities}{H}
\end{figure}
\par\noindent\rule{\textwidth}{0.4pt}

    
\pagebreak


\vspace{5mm}
\begin{center}
{\Large\textbf{Source code}}
\end{center}
\par\noindent\rule{\textwidth}{0.4pt}


\begin{FlushLeft}


\begin{verbatim}

#include <stdio.h>
#include <stdlib.h>
//Global variable Declaration

int Number_of_candidates_enrolled = 0;

//i is a variable which keeps the track of number of candidates
int i = 0;//The tracking variable or the index variable.


// Structure to store the candidate's different details
struct candidate_info {
    char   name[50];
    int    roll;
    int    chem_marks;
    int    phy_marks;
    int    math_marks;
    char   father_name[20];
    char   mother_name[20];
    int    age;
    char   dob[10];
    int    category_number;
    int    all_india_rank;
    char   district[10];
    int    jee_rank;
    double percentile;
};
struct candidate_info c[50];



//All the functions are declared here so that any function can access any funcion irrespective of any sequence.
void    display();
double  percentile_calculator(int rank);
int     jee_log();
void    find_candidate();
float   total(float x,float y,float z);
void    advance_qualification();
void    take_candidate_info();
void    record_the_entry();
void    take_candidate_info();



//1. Function to add the candidates along with all thier details
//It takes differnt input details of the candidate
void take_candidate_info()
{
    printf("\n\t\t\t\tEnter the name of the candidate : ");
    fflush(stdin);//It is used to avoid errors coming while taking the inputs
    gets(c[i].name);//input the name

    printf("\n\t\t\t\tEnter the father name of the candidate: ");
    fflush(stdin);
    gets(c[i].father_name);//input the father name of the candidate

    printf("\n\t\t\t\tEnter the mother name of the candidate: ");
    fflush(stdin);
    gets(c[i].mother_name);////input the mother name of the candidate

    printf("\n\t\t\t\tEnter the date of birth of the candidate in the format (date-month-year): ");
    fflush(stdin);
    gets(c[i].dob);//input the name date of birth of the candidate

    printf("\n\t\t\t\tEnter age: ");
    scanf("%d", &c[i].age);//input the age

    printf("\n\t\t\t\tEnter the enrollment number of the candidate starting from 1 : ");
    scanf("%d", &c[i].roll);//input the roll number of the candidate

    printf("\n\t\t\t\tEnter the chemistry marks of the candiate :");
    scanf("%d", &c[i].chem_marks);//input the chemistry marks

    printf("\n\t\t\t\tEnter the physics marks of the candiate :");
    scanf("%d", &c[i].phy_marks);//input the physics marks

    printf("\n\t\t\t\tEnter the maths marks of the candiate :");
    scanf("%d", &c[i].math_marks);//input the maths marks

    printf("\n\t\t\t\tEnter the home town:");
    scanf("%s", &c[i].district);//input the district

    printf("\n\t\t\t\tEnter the all india rank of the candidate :");
    scanf("%d", &c[i].all_india_rank);//input the all India rank
    
    printf("\n\t\t\t\tEnter the category of the candidate:");
    printf("\n\t\t\t\t1--->General");
    printf("\n\t\t\t\t2--->OBC");
    printf("\n\t\t\t\t3--->SC");
    printf("\n\t\t\t\t4--->ST  : ");
    scanf("%d", &c[i].category_number);//input the category                              
    printf("\n");
    display();//display function is called here.

    c[i].percentile = percentile_calculator(c[i].all_india_rank);
    i++; //This is a update index statement so that the information of the next candidate gets stored in the 
    // c[i+1].

}




//2. A simple total function which takes 3 inputs and return the sum of all
float total(float x,float y,float z){
    float sum = x+y+z;
    return sum;
}




//3.Function to display the details after adding the candidate info
//It is specifically designed with correct numbers of the identations to give a table like look.
void display(){
    printf("\n------------------------------------------------------------------------------------\n");
    printf("|                               JOINT ENTRANCE EXAMINATION                           |\n");
    printf("|------------------------------------------------------------------------------------\n");
    printf("|Name: %s\tCategory: %dth\tRoll Number: %d\n", c[i].name, c[i].category_number, c[i].roll);
    printf("|Father name: %s\t JEE rank :%d\n", c[i].father_name,c[i].all_india_rank);
    printf("|Mother name: %s\n",  c[i].mother_name);
    printf("|Date of birth: %s\n",  c[i].dob);
    printf("|District: %s\n",  c[i].district);
    printf("|------------------------------------------------------------------------------------\n");
    printf("|\tSUBJECTS      \tMAX MARKS\tMIN MARKS NEEDED\tTHEORY MARKS\n");
    printf("|------------------------------------------------------------------------------------\n");
    printf("|\tPhysics      \t  100      \t   33       \t\t     %d\n",c[i].phy_marks);
    printf("|\tMaths       \t  100      \t   33       \t\t     %d\n", c[i].math_marks);
    printf("|\tChemistry      \t  100      \t   33       \t\t     %d\n", c[i].chem_marks);
    printf("|------------------------------------------------------------------------------------\n");
    printf("|\t              \t  300    |    GRAND TOTAL\t    %f\n", total(c[i].phy_marks,c[i].math_marks,c[i].phy_marks));
    printf("|Percentile: %lf\n", percentile_calculator(c[i].all_india_rank) );
    printf("|------------------------------------------------------------------------------------\n");
}





// 4.Function to find the candidate by the roll number
//This function will take a roll number as input and return all the details of that candidate.
void find_candidate()
{
    int x;
    int flag;
    int position;
    fflush(stdin);
    printf("Enter the Roll Number"
           " of the candidate\n");
    scanf("%d", &x);
    for (int j = 0; j < 50; j++)
    {
        if(x  == c[j].roll){
            flag =1;
            position = j;
        }
    }
    //The above segment finds the index variable of the array of the structure used 
    //as per the roll number entered by the user 

    //Below segment comes into the role after the successful indentification of the index variable.
    //This segments prints all the details of that candidate
    if (flag == 1){
         printf("\n\t\t\tName of the candidate is : %s",c[position].name);
         printf("\n\t\t\tFather's Name of the candidate is : %s",c[position].father_name);
         printf("\n\t\t\tMother's Name of the candidate is : %s",c[position].mother_name);
         printf("\n\t\t\tPhysics marks of the candidate are : %d",c[position].phy_marks);
         printf("\n\t\t\tMaths marks of the candidate are : %d",c[position].math_marks);
         printf("\n\t\t\tChemistry marks of the candidate are : %d ",c[position].chem_marks);
         printf("\n\t\t\tAge of the candidate is : %d ",c[position].age);
         printf("\n\t\t\tThe date of birth of the candidate is : %s",c[position].dob);
         printf("\n\t\t\tRank of the candidate is : %d",c[position].all_india_rank);
         printf("\n\t\t\tPercentile of the candidate aacording to the number is : %lf",c[position].percentile);
         
    }   
}





//5.Function to modify the candidate details
//This function also takes roll number as inputs and is used to modify the details of that student. 
void modify(){
    int x;
    int flag;
    int position;
    fflush(stdin);
    printf("Enter the Roll Number of the candidate\n");
    scanf("%d", &x);
    for (int j = 0; j < 50; j++)
    {
        if(x  == c[j].roll){
            flag =1;
            position = j;
        }
    }
    //The above segment finds the index variable of the array of the structure used 
    //as per the roll number entered by the user 
    
    //Below segment comes into the role after the successful indentification of the index variable.
    //This segments reassigns the details of that candidate
    if (flag == 1){
    printf("\n\t\t\t\tEnter the name of the candidate : ");
    fflush(stdin);
    gets(c[position].name);
    printf("\n\t\t\t\tEnter the father name of the candidate: ");
    fflush(stdin);
    gets(c[position].father_name);
    printf("\n\t\t\t\tEnter the mother name of the candidate: ");
    fflush(stdin);
    gets(c[position].mother_name);
    printf("\n\t\t\t\tEnter the date of birth of the candidate in the format (date-month-year): ");
    fflush(stdin);
    gets(c[position].dob);
    printf("\n\t\t\t\tEnter age: ");
    scanf("%d", &c[position].age);
    printf("\n\t\t\t\tEnter the chemistry marks of the candiate :");
    scanf("%d", &c[position].chem_marks);
    printf("\n\t\t\t\tEnter the physics marks of the candiate :");
    scanf("%d", &c[position].phy_marks);
    printf("\n\t\t\t\tEnter the maths marks of the candiate :");
    scanf("%d", &c[position].math_marks);
    printf("\n\t\t\t\tEnter the home town:");
    scanf("%s", &c[position].district);
    printf("\n\t\t\t\tEnter the all india rank of the candidate :");
    scanf("%d", &c[position].all_india_rank);
         
    }
       
}





//6.Function for entering jee conduction details
//also this function throws a brief introduction about the JEE examination.
int jee_log(){
    printf("\n\t\t\t\tJEE or Joint Entrance Examination\n");
    printf("The Joint Entrance Examination (JEE) is an engineering entrance"
    " assessment conducted for admission to various  engineering colleges in India."
    "It is constituted by two different examinations: the JEE-Main and the JEE-Advanced.\n\n ");
    printf("\n\t\t\t\tEnter the number of students attempted jee this year :");
    scanf("%d",&Number_of_candidates_enrolled);
    return  Number_of_candidates_enrolled;
}




    
//7.This function checks the qualification for advance:
//The function takes the roll number as inputs and returns advanced qualification status on the basis of category
//of the candidate. 
void advance_qualification(){
    int  x;
    int  flag;  //Used to carry forward the process after the identification to the trackiing the variable
    int  position;
    fflush(stdin);
    printf("Enter the Roll Number"
           " of the candidate\n");
    scanf("%d", &x);
    for (int j = 0; j < 50; j++)
    {
        if(x  == c[j].roll){
            flag =1;
            position = j;
        }
    }
    //The above segment finds the index variable of the array of the structure used 
    //as per the roll number entered by the user 

    if (flag == 1) 
        {printf("As per the category criteria \n");

        //checking the category of the candidate and then examining the cutoff marks for the advance 
        //qualification as per the category.
        if(c[position].category_number ==1 ){
            printf("The candidate belong to general category \n");
            if(c[position].percentile>=90){
                printf("The candidate has qualified for jee advanced \n");
            }
        }

        else if(c[position].category_number ==2 ){
            printf("The cadidate belongs to the OBC category \n");
            if(c[position].percentile>=83){
                printf("The candidate has qualified for jee advanced \n");
            }
        }
        else if(c[position].category_number ==3 ){
            printf("The cadidate belongs to the SC category \n");
            if(c[position].percentile>=75){
                printf("The candidate has qualified for jee advanced \n");
            }
        }
        else if(c[position].category_number ==4 ){
            printf("The cadidate belongs to the ST category \n");
            if(c[position].percentile>=60){
                printf("The candidate has qualified for jee advanced \n");
            }
        }
        else{
            printf("Enter the correct category \n");
        }
    }
}


   



//8.Program to find the percentile of the candidate
//the function uses a simple algorithm which converts the rank to the percentile
double percentile_calculator(int rank){
    double percentile;
    if((Number_of_candidates_enrolled !=0)|| (Number_of_candidates_enrolled >= rank))
    {
        percentile = (double)(Number_of_candidates_enrolled - rank)*100/Number_of_candidates_enrolled;
    }
    else{
        printf("Please enter the correct number of the students attempted the exam \n");
    }
    return percentile;
}

//9.The below function uses file handling and prints the data of the candidate into a seperate text file(emp.txt)
void record_the_entry(){
    FILE *fptr; 
    fptr = fopen("emp.txt", "a+");/*  open for writing */  
    if (fptr == NULL)  
    {  
        printf("File does not exists \n");
    }   
    fprintf(fptr, "Id= %d\n", i);   
    fprintf(fptr, "Name= %s\n", c[i-1].name);
    fprintf(fptr, "Age= %d\n", c[i-1].age); 
    fprintf(fptr, "Rank= %d\n", c[i-1].all_india_rank);
    fprintf(fptr, "Percentile = %lf\n", c[i-1].percentile);
    fprintf(fptr, "dob= %d\n", c[i-1].all_india_rank);
    fprintf(fptr, "Mother name= %s\n", c[i-1].mother_name);
    fprintf(fptr, "District= %s\n", c[i-1].district);
    fprintf(fptr, "Roll number= %d\n", c[i-1].roll);  
    fclose(fptr); 
    printf("Data is successful is loaded \n"); 
}




//10.main function
int main(){
    int arbitrary_rank;
    int loop_terminater = 1;
    int option;

    //A catchy display
    printf("\n\n\t\t\t\t Joint Entrance Examination (2022)");
    printf("\n\n\t\t\t\t\t\t   NTA \n");

    // The below segment redirects the user to the jee_log function to fetch some basic data needed for the 
    //program running.
    printf("\n\n\t\t\t\t\t   JEE Information\n");
    jee_log();

    //A while loop is introduced over here to basically put the program in an unending loop
    //So that as far as user wants the program runs
    while(loop_terminater == 1)
    {
        printf("\n\n");
        printf("\n\n\t\t\t\t   Select from the given options");
        printf("\n\n\t\t\t\t1. Add Student");
        printf("\n\n\t\t\t\t2. Find the candidate by the roll number");
        printf("\n\n\t\t\t\t3. Modify Student");
        printf("\n\n\t\t\t\t4. Record the entry int the text file");
        printf("\n\n\t\t\t\t5. check elegibility for the jee advance");
        printf("\n\n\t\t\t\t6. Criterion and information");
        printf("\n\n\t\t\t\t7. Rank to percentile convertor");
        printf("\n\n\t\t\t\t8. Logout");
        printf("\n\n\t\t\t\tEnter Your Option :--> ");
        scanf("%d",&option);
        fflush(stdin);
        

        //Switch case is used here to give the user with various options each leading to a different function 
        //and ultimately to the different use and utility.
        switch (option)
        {

        //This case redirects the user to take_candidate_info function followed by the display function
        
        case 1:
            take_candidate_info();
            break;

        //This case redirects the user to find_candidate function
        case 2:
            find_candidate();
            break;

        //This case redirects the user to modify function
        case 3:
            modify();
            break;

        //This case redirects the user to record_the_entry function
        case 4:
            record_the_entry();         
            break;
        
        //This case redirects the user to advance_qualification function
        case 5:
            advance_qualification();
            break;
        
        //This case redirects the user to jee_log function
        case 6:
            jee_log();
            break;
        
        //This case will take a arbitrary rank and by using the percentile_calculator function will convert it to
        //the percentile
        case 7:
            printf("Enter the rank: ");
            scanf("%d",&arbitrary_rank);
            printf("The percentile is : %lf",percentile_calculator(arbitrary_rank));
            
            break; 
        // The objective of the loop terminater is to end the while rule and ultimately the program
        case 8:
            loop_terminater = 0;
            break;
        
        
        default:
            break;
        }
        
    }
    return 0;
} 
} 



    
\end{verbatim}
\end{FlushLeft}
\vspace{5mm}
\par\noindent\rule{\textwidth}{1 pt}
\pagebreak

\vspace{5mm}
\begin{center}
{\Large\textbf{Output Screenshots }}
\par\noindent\rule{\textwidth}{1 pt}
\end{center}\\

\begin{figure}
    \begin{center}
        \includegraphics[height=15cm, width=20cm,inner]{image_1.png}
    \end{center}
    \caption{Initial Display}
    \label{fig:my_label}
\end{figure}
\\
\begin{figure}
    \centering
    \includegraphics[height=15cm, width=20cm]{image_2.png}
    \caption{Display Function}
\end{figure}
\\
\begin{figure}
    \centering
    \includegraphics[height=15cm, width=20cm]{image_3.png}
    \caption{find candidate function}
\end{figure}

\\
\begin{figure}
    \centering
    \includegraphics[height=15cm, width=20cm]{image_4.png}
    \caption{find candidate function}
\end{figure}

\\
\begin{figure}
    \centering
    \includegraphics[height=15cm, width=20cm]{image_5.png}
    \caption{find candidate function}
\end{figure}

\\
\begin{figure}
    \centering
    \includegraphics[height=10cm, width=10cm]{image_6.png}
    \caption{find candidate function}
\end{figure}

\\
\begin{figure}
    \centering
    \includegraphics[height=10cm, width=10cm]{image_7.png}
    \caption{A text file is created having the report of previous entries }
\end{figure}

\\
\begin{figure}
    \centering
    \includegraphics[height=15cm, width=20cm]{image_8.png}
    \caption{Using the advance qualification function}[H]
\end{figure}

\\
\begin{figure}
    \centering
    \includegraphics[height=15cm, width=20cm]{image_9.png}
    \caption{percentile converter in work}[H]
\end{figure}
\pagebreak

\\
\begin{figure}
    \centering
    \includegraphics[height=15cm, width=20cm]{image_10.png}
    \caption{again added a candidate information}[H]
\end{figure}



\end{document}

